\documentclass{article}
\usepackage[utf8]{inputenc}
\usepackage[italian]{babel}
\usepackage{geometry}
\usepackage{amsfonts} 
\usepackage{ccicons}
\usepackage{url}
\usepackage{hyperref}
\usepackage{amsmath}
\usepackage{xfrac}
\usepackage{cellspace}
\usepackage{setspace}
\setlength\cellspacetoplimit{4pt}
\setlength\cellspacebottomlimit{4pt}

\geometry{left = 2 cm, right = 2 cm, bottom = 2 cm, top = 2 cm}
\setlength{\parindent}{0em}
\hypersetup{
	colorlinks=true,
	urlcolor=blue,
    colorlinks,
    linkcolor={black},
    citecolor={black}
}

\pagestyle{plain}
\pagenumbering{gobble}

\title{\Huge Formulario di Elettronica dello stato solido}
\author{\LARGE Lorenzo Rossi}
\date{\LARGE Anno Accademico 2020/2021}


\begin{document}
\maketitle

\vspace{18em}

\large
\begin{doublespacing}\hypersetup{
    urlcolor=black,
  }\urlstyle{same}
  \centerline{Email: \href{mailto://lorenzo14.rossi@mail.polimi.it}{lorenzo14.rossi@mail.polimi.it}}
  \centerline{GitHub: \url{https://github.com/lorossi}}

  \vspace{18em}
  \centerline{Quest'opera è distribuita con Licenza Creative Commons Attribuzione}
  \centerline{Non commerciale 4.0 Internazionale \ccbynceu}
  \centerline{Versione aggiornata al 15/04/2021}
\end{doublespacing}
\newpage


\pagenumbering{roman}
\tableofcontents
\clearpage
\pagenumbering{arabic}
\newpage

\section{Riguardo al formulario}
Quest'opera è distribuita con Licenza Creative Commons - Attribuzione Non commerciale 4.0 Internazionale \ccbynceu \newline
Questo formulario verrà espanso (ed, eventualmente, corretto) periodicamente fino a fine corso (o finché non verrà ritenuto completo). \newline
Link repository di GitHub: \url{} \newline
L'ultima versione può essere scaricata direttamente cliccando \href{https://github.com/lorossi/formulario-stato-solido/raw/master/formulario-elettronica-dello-stato-solido.pdf}{su questo link.}


\section{Struttura cristallina}
\begin{itemize}
  \item Packing factor \( PF = \dfrac{ \sfrac{4}{3} \cdot \pi r^3}{a^3} \)
  \item Densità del reticolo \( l = \dfrac{\text{n} ^ \circ \text{atomi / cella}}{\text{area cella}} \)
  \item Interferenza del passo reticolare (diffrazione alla Bragg) \(  2a \sin \theta = n \lambda \) con \( n \) ordine di diffrazione
\end{itemize}

\vspace{1cm}
\renewcommand{\arraystretch}{2}
\begin{tabular}{|c|c|c|}
  \hline
  \textbf{Struttura} & \textbf{Metalli che la presentano in natura} & \textbf{Packing Factor}                  \\
  \hline
  Cubico             & Po                                           & \(\frac{\pi}{6} \approx 0.52 \)          \\
  \hline
  GBB                & Cr, Fe, Mo, Ta                               & \(\pi \frac{\sqrt{3}}{8} \approx 0.68 \) \\
  \hline
  FCC                & Ag, Au, Cu, Ni, Pb                           & \(\pi \frac{\sqrt{2}}{6} \approx 0.74 \) \\
  \hline
\end{tabular}
\renewcommand{\arraystretch}{1}
\vspace{1cm}

\subsection{Indici di Miller}
\textbf{Ipotesi:} il piano interseca in \( \{m, n, 0\} \)
\begin{itemize}
  \item Indici di Miller \( \{n, m, 0\} \)
  \item Distanza interplanare \( d = \dfrac{a}{\sqrt{n^2+m^2}} \)
\end{itemize}

\section{Radiazione di corpo nero}
\begin{itemize}
  \item Legge di Wien \( \lambda_{ma} \cdot T = K_{\textnormal{wien}} \)
  \item Legge di Stefan \( \displaystyle \int_{0}^{\inf} R_T d \nu = \sigma T ^ 4 \)
\end{itemize}
\subsection{Cavità di corpo nero all'equilibrio}
\subsubsection{Cavità monodimensionale}
\begin{itemize}
  \item Lunghezze d'onda permesse \( a = n \dfrac{\lambda}{2} \)
  \item Frequenze permesse \( \nu = \dfrac{c}{2a} n \) con \( n \) intero e non nullo
  \item Free spectral range FSR \( = \dfrac{c}{2a} \)
\end{itemize}

\section{Onde e particelle}
\subsection{Onde}
\begin{itemize}
  \item Frequenza / lunghezza d'onda \( \nu = \dfrac{c}{\lambda} \)
  \item Energia associata ad un'onda \( E = h \nu \)
  \item Vettore d'onda \( k = \dfrac{2 \pi}{h} \)
  \item Velocità di fase \( v_f = \dfrac{d \omega}{d t}\)
  \item Velocità di gruppo \( v_g = \dfrac{\partial \omega}{\partial k} = \dfrac{\hbar k}{m} \)
\end{itemize}

\subsection{Particelle}
\begin{itemize}
  \item Energia \( E = E_k + U \)
        \begin{itemize}
          \item Energia cinetica \( E_k = \dfrac{1}{2} m v ^ 2 \)
          \item Principio di equipartizione dell'energia, particella con \( l \) gradi di libertà: \( E_k = \dfrac{l}{2} k T \)
          \item Energia potenziale di una particella in un potenziale \( V \): \( U = qV \)
        \end{itemize}
  \item Relazione di De Broglie \( \lambda = \dfrac{h}{p} \), \( p = \hbar k \)
  \item Relazione di dispersione \( E = \dfrac{h ^ 2 k ^ 2}{2 m} \)
  \item Vettore d'onda \( k = \dfrac{\sqrt{2mE}}{\hbar} \)
  \item Lunghezza d'onda \( \lambda = \dfrac{h}{\sqrt{2mE}} \)
\end{itemize}

\section{Meccanica quantistica}
\begin{itemize}
  \item Principio di indeterminazione di Heisenberg \( \Delta x \Delta p \geq \dfrac{\hbar}{2} \)
  \item Equazione di Schrödinger \( i \hbar \dfrac{\partial \Psi}{\partial t} (x, t) = \hat{H} \Psi (x,t) \)
  \item Flusso quantistico \( J = \dfrac{\hbar k}{m} \left| \Psi \right| ^ 2 \)
\end{itemize}

\subsection{Operatori}
\begin{itemize}
  \item Operatore Hamiltoniano \( - \dfrac{\hbar ^ 2}{2mi} \dfrac{\partial ^ 2}{\partial x ^ 2} + V \)
  \item Operatore quantità di moto (momento) \( \hat{p} = -i \hbar \dfrac{\partial}{\partial x} \)
  \item Operatore energia cinetica \( \hat{E}_{tot} = -i \dfrac{\hbar ^ 2}{2m} \dfrac{\partial ^ 2}{\partial t ^ 2} \)
  \item Operatore energia totale \( \hat{E}_k = i \hbar \dfrac{\partial }{\partial t} \)
  \item Operatore potenziale \( \hat{V} = V \)
  \item Commutatore \( H = [\hat{A}, \hat{B}] = \hat{A}\hat{B} - \hat{B}\hat{A} = \hat{C} \)
\end{itemize}

\subsection{Tunneling}
\begin{itemize}
  \item Probabilità di tunneling \( \left| T \right| ^ 2 \approx 16 \left( \dfrac{\alpha k}{\alpha^2 + k^2} \right) ^ 2  \exp{ \left\{ -2 \alpha a \right\} } \approx \exp{\left\{-2 \alpha a\right\}} \)
  \item Tempo medio di tunneling \( \langle t \rangle = \dfrac{t_{a/r}}{p_t} = \dfrac{2 a}{v p_{\text{tun}}} \)
  \item Approssimazione WKB
        \begin{itemize}
          \item Probabilità \( \left| T \right| ^ 2 = P_T = \exp{ \left\{ -2 \alpha a \right\} } \)
          \item Penetrazione media \( x_p = \dfrac{\hbar}{\sqrt{2 m (V_0 - E)}} = \dfrac{1}{\alpha} \)
          \item \textbf{L'approssimazione è valida se e solo se} \( \alpha a \gg 1 \)
          \item \textbf{Caso particolare: barriera triangolare} \( P_T = \exp\left\{ - \dfrac{4}{3} \dfrac{\sqrt{2 m}}{\hbar}  \dfrac{\Phi ^ {\sfrac{3}{2}}}{qF} \right\} \)
        \end{itemize}
  \item Approssimazione di Follower-Nonditeim
        \begin{itemize}
          \item
        \end{itemize}
\end{itemize}

\subsection{Incidenza}
\begin{itemize}
  \item Coefficiente di riflessione \( R = \left( \dfrac{k_1 - k_2}{k_1 + k_2} \right) ^ 2 \)
  \item Coefficiente di trasmissione \( T = \left( \dfrac{2 k_1}{k_1 + k_2} \right) ^ 2 = 1 - R ^ 2 \)
\end{itemize}

\subsection{Buca di potenziale}

\subsubsection{A pareti infinite}
\begin{itemize}
  \item Autovalori \( E_n = \dfrac{h^2}{8 m a ^2} n^2 \)
\end{itemize}

\subsubsection{A pareti finite}
\begin{itemize}
  \item Funzioni pari \( \tan\left( \dfrac{a}{2 \hbar} \sqrt{2mE} \right) = \sqrt{\dfrac{V_0 - E}{E}} \)
  \item Funzioni dispari \( \tan\left( \dfrac{a}{2 \hbar} \sqrt{2mE} \right) = - \sqrt{\dfrac{E}{V_0 - E}} \)
  \item La soluzione delle equazioni avviene per via grafica
\end{itemize}

\subsubsection{Parabolica}
\begin{itemize}
  \item Profilo di potenziale \( U = \dfrac{1}{2} \alpha x ^ 2 \)
  \item Pulsazione caratteristica \( \omega = \sqrt{\dfrac{\alpha}{m}} \)
  \item Autovalori \( E_n = ( n + \dfrac{1}{2} ) \hbar \omega \)
\end{itemize}

\subsubsection{Coppie di buche}
\begin{itemize}
  \item Funzione degli autovalori \( \tan\left( k \dfrac{a}{2} \right) = - \dfrac{\hbar ^2 k}{m U_0} \)
  \item Proporzionalità della ddp \( | \psi | ^ 2 \propto \cos \left( \dfrac{E_2 - E_1}{\hbar} t \right) = \cos \left( 2 \pi \dfrac{E_2 - E_1}{h} t \right)  \)
        \begin{itemize}
          \item Oscillazione degli autovalori \( \omega = \dfrac{E_2 - E_1}{\hbar} \)
          \item Frequenza degli autovalori \( \nu = \dfrac{E_2 - E_1}{h} \)
        \end{itemize}
\end{itemize}

\end{document}

\documentclass{article}
\usepackage[utf8]{inputenc}
\usepackage[italian]{babel}
\usepackage{geometry}
\usepackage{amsfonts} 
\usepackage{ccicons}
\usepackage{url}
\usepackage{hyperref}
\usepackage{amsmath}
\usepackage{xfrac}
\usepackage{cellspace}
\usepackage{setspace}
\usepackage{lmodern}
\setlength\cellspacetoplimit{4pt}
\setlength\cellspacebottomlimit{4pt}

\geometry{left = 2 cm, right = 2 cm, bottom = 2 cm, top = 2 cm}
\setlength{\parindent}{0em}
\hypersetup{
	colorlinks=true,
	urlcolor=blue,
    colorlinks,
    linkcolor={black},
    citecolor={black}
}

\pagestyle{plain}
\pagenumbering{gobble}

\title{\Huge Formulario di Elettronica dello stato solido}
\author{\LARGE Lorenzo Rossi}
\date{\LARGE Anno Accademico 2020/2021}


\begin{document}
\maketitle

\vspace{18em}

\large
\begin{doublespacing}\hypersetup{
    urlcolor=black,
  }\urlstyle{same}
  \centerline{Email: \href{mailto://lorenzo14.rossi@mail.polimi.it}{lorenzo14.rossi@mail.polimi.it}}
  \centerline{GitHub: \url{https://github.com/lorossi}}

  \vspace{18em}
  \centerline{Quest'opera è distribuita con Licenza Creative Commons Attribuzione}
  \centerline{Non commerciale 4.0 Internazionale \ccbynceu}
  \centerline{Versione aggiornata al 18/05/2021}
\end{doublespacing}
\newpage


\pagenumbering{roman}
\tableofcontents
\clearpage
\pagenumbering{arabic}
\newpage

\section{Riguardo al formulario}
Quest'opera è distribuita con Licenza Creative Commons - Attribuzione Non commerciale 4.0 Internazionale \ccbynceu \newline
Questo formulario verrà espanso (ed, eventualmente, corretto) periodicamente fino a fine corso (o finché non verrà ritenuto completo). \newline
Link repository di GitHub: \url{https://github.com/lorossi/formulario-stato-solido} \newline
L'ultima versione può essere scaricata direttamente cliccando \href{https://github.com/lorossi/formulario-stato-solido/raw/master/formulario-elettronica-dello-stato-solido.pdf}{su questo link.} \newline
In questo formulario ho cercato prima di tutto di mettere le formule importanti per la risoluzione degli esercizi, preferendole a quelle utili alla comprensione della materia.

\newpage

\section{Richiami di Base}
\subsection{Serie}
\begin{itemize}
  \item Serie geometrica \( \displaystyle  s_n = \sum_{n=0}^{+\inf} q^n = 1 + q + q^2 + ... \) converge a \( s_n = \dfrac{1}{1-q} \) se \( |q| < 1 \)
  \item Serie armonica \( \displaystyle s_n = \sum_{n=0}^{+\inf} \dfrac{1}{n ^ \alpha} \) converge se \( \alpha > 1 \)
\end{itemize}

\subsection{Elettromagnetismo}
\begin{itemize}
  \item Forza \( |F| = \dfrac{|V|}{|x|} \)
  \item Forza vettoriale \( \vec{F} = - \vec{\nabla} V = -\dfrac{1}{q} \vec{\nabla} U \)
  \item Campo elettrico \( E = -q V \)
  \item Energia \( \delta E = q F \delta x \), \( \dfrac{\partial E}{\partial T} = q F v_g  \)
\end{itemize}

\newpage

\section{Struttura cristallina}
\begin{itemize}
  \item Packing factor \( PF = \dfrac{ \sfrac{4}{3} \cdot \pi r^3}{a^3} \)
  \item Densità del reticolo \( l = \dfrac{\text{n} ^ \circ \text{atomi / cella}}{\text{area cella}} \)
  \item Interferenza del passo reticolare (diffrazione alla Bragg) \(  2a \sin \theta = n \lambda \) con \( n \) ordine di diffrazione
\end{itemize}

\subsection{Packing factor}
\vspace{1cm}
\renewcommand{\arraystretch}{2}
\begin{center}
  \begin{tabular}{|c|c|c|}
    \hline
    \textbf{Struttura} & \textbf{Metalli che la presentano in natura} & \textbf{Packing Factor}                  \\
    \hline
    Cubico             & Po                                           & \(\frac{\pi}{6} \approx 0.52 \)          \\
    \hline
    GBB                & Cr, Fe, Mo, Ta                               & \(\pi \frac{\sqrt{3}}{8} \approx 0.68 \) \\
    \hline
    FCC                & Ag, Au, Cu, Ni, Pb                           & \(\pi \frac{\sqrt{2}}{6} \approx 0.74 \) \\
    \hline
  \end{tabular}
\end{center}
\renewcommand{\arraystretch}{1}
\vspace{1cm}

\subsection{Indici di Miller}
\textbf{Ipotesi:} il piano interseca in \( \{m, n, 0\} \)
\begin{itemize}
  \item Indici di Miller \( \{n, m, 0\} \)
  \item Distanza interplanare \( d = \dfrac{a}{\sqrt{n^2+m^2}} \)
\end{itemize}

\newpage

\section{Radiazione di corpo nero}
\begin{itemize}
  \item Legge di Wien \( \lambda_{p} \cdot T = c_{\textnormal{wien}} \)
  \item Legge di Stefan-Boltzman \( \displaystyle \int\limits_{0}^{\inf} R_T d \nu = \sigma T ^ 4 \)
  \item Potenza emessa dal corpo nero \( P = \sigma T^4 A = R A \), con \(A\) area della superficie del corpo
\end{itemize}
\subsection{Cavità di corpo nero all'equilibrio, monodimensionale}
\begin{itemize}
  \item Lunghezze d'onda permesse \( a = n \dfrac{\lambda}{2} \)
  \item Frequenze permesse \( \nu = \dfrac{c}{2a} n \) con \( n \) intero e non nullo
  \item Free spectral range \( \text{FSR} = \nu_n - \nu_{n-1} = \dfrac{c}{2a} \)
\end{itemize}

\newpage

\section{Onde e particelle}
\subsection{Onde}
\begin{itemize}
  \item Frequenza / lunghezza d'onda \( \nu = \dfrac{c}{\lambda} \)
  \item Energia associata ad un'onda \( E = h \nu = \hbar \omega \)
  \item Vettore d'onda \( k = \dfrac{2 \pi}{h} \)
  \item Velocità di fase \( v_f = \dfrac{d \omega}{d t}\)
  \item Velocità di gruppo \( v_g = \dfrac{\partial \omega}{\partial k} = \dfrac{1}{\hbar} \dfrac{\partial E}{\partial k} =\dfrac{\hbar k}{m} \)
\end{itemize}

\subsubsection{Pacchetti d'onda}
\begin{itemize}
  \item Formula generale \( \Psi(x, t) = \int{g(k) e^{i (kx- \omega t)} dk}  \)
  \item Densità di probabilità \( \left| \Psi(x, t) \right| ^ 2 = \exp\left\{-\dfrac{(x-v_g t) ^ 2}{2 \alpha (1 + \sfrac{\beta^2 t^2}{\alpha^2})}\right\} \sqrt{\dfrac{\pi^2}{\alpha^2 + \beta^2 t^2}} \)
  \item Deviazione standard \( \sigma_x (t) = \sqrt{ \dfrac{\alpha^2 + \beta^2 t^2}{\alpha} } \)
  \item Pacchetto gaussiano:
        \begin{itemize}
          \item Velocità \( v_g = \dfrac{\partial \omega}{\partial k} \)
          \item Dispersione \(\beta = \dfrac{1}{2} \dfrac{\partial ^ 2 E}{\partial ^ 2 k} \)
          \item Oscillazione \( \omega = \omega_0 + v_g \cdot (k - k_0) + \beta \cdot (k - k_o) ^ 2 \)
          \item Il picco si sposta con \( v = v_g \)
        \end{itemize}
\end{itemize}

\subsection{Particelle}
\begin{itemize}
  \item Energia \( E = E_k + U \)
        \begin{itemize}
          \item Energia cinetica \( E_k = \dfrac{1}{2} m v ^ 2 = \dfrac{h ^ 2}{2 m \lambda }\)
          \item Principio di equipartizione dell'energia, particella con \( l \) gradi di libertà: \( E_k = \dfrac{l}{2} k T \)
          \item Energia potenziale di una particella in un potenziale \( V \): \( U = qV \)
        \end{itemize}
  \item Relazione di De Broglie \( \lambda = \dfrac{h}{p} \), \( p = \hbar k \)
  \item Relazione di dispersione \( E = \dfrac{h ^ 2 k ^ 2}{2 m} \)
  \item Vettore d'onda \( k = \dfrac{\sqrt{2mE}}{\hbar} \)
  \item Lunghezza d'onda \( \lambda = \dfrac{h}{\sqrt{2mE}} \)
\end{itemize}

\newpage

\section{Meccanica quantistica}
\begin{itemize}
  \item Principio di indeterminazione di Heisenberg \( \Delta x \Delta p \geq \dfrac{\hbar}{2} \)
  \item Equazione di Schrödinger \( i \hbar \dfrac{\partial \Psi}{\partial t} (x, t) = \hat{H} \Psi (x,t) \)
  \item Flusso quantistico \( J = \dfrac{\hbar k}{m} \left| \Psi \right| ^ 2 \)
\end{itemize}

\subsection{Teorema di Bloch}
\textbf{Ipotesi:}
\begin{itemize}
  \item Struttura reticolare con passo \( a \)
  \item Il potenziale è periodico \( V(x+a) = V(x) \)
  \item La funzione d'onda si ripete a meno di un fattore di fase \( \psi(x+a) = \psi(x) e^{ i k  a} \)
  \item La densià di probabilità è periodica \( \left| \psi(x+a) \right|^2 = \left| \psi(x)  \right|^2 \)
\end{itemize}
\textbf{Allora:} \( \left| \psi(x) \right| = u_k(x) e^{ -ikx } \) con \( u_k(x) \) funzione di Bloch \textit{(periodica)}, quindi \( u_k (x+a) = u_k(x).\) e \( e^{ -ikx } \) inviluppo. \newline \textit{Normalmente è costruita da \(\sin^2\) o \(\cos^2\), con i massimi in corrispondenza dei centri delle barriere}. \newline
Inoltre \( \psi(x+a) = u_k (x+a) e^{ikx} e^{ika} \), con \( e^{ika} \) sfasamento.

\subsection{Operatori}
\begin{itemize}
  \item Operatore Hamiltoniano \( \hat{H} = - \dfrac{\hbar ^ 2}{2mi} \dfrac{\partial ^ 2}{\partial x ^ 2} + V \)
  \item Operatore quantità di moto (momento) \( \hat{p} = -i \hbar \dfrac{\partial}{\partial x} \)
  \item Operatore energia cinetica \( \hat{E}_{tot} = -i \dfrac{\hbar ^ 2}{2m} \dfrac{\partial ^ 2}{\partial t ^ 2} \)
  \item Operatore energia totale \( \hat{E}_k = i \hbar \dfrac{\partial }{\partial t} \)
  \item Operatore potenziale \( \hat{V} = V \)
  \item Commutatore \( H = [\hat{A}, \hat{B}] = \hat{A}\hat{B} - \hat{B}\hat{A} = \hat{C} \)
        \begin{itemize}
          \item Se \( \hat{C} = 0 \), allora i due operatori \textit{commutano}.
        \end{itemize}
\end{itemize}

\subsection{Tunneling}
\begin{itemize}
  \item Probabilità di tunneling \( \left| T \right| ^ 2 \approx 16 \left( \dfrac{\alpha k}{\alpha^2 + k^2} \right) ^ 2  \exp{ \left\{ -2 \alpha a \right\} } \approx \exp{\left\{-2 \alpha a\right\}} \)
  \item Tempo medio di tunneling \( \langle t \rangle = \dfrac{t_{a/r}}{p_t} = \dfrac{2 a}{v p_{\text{tun}}} \)
  \item Approssimazione WKB
        \begin{itemize}
          \item Probabilità \( \left| T \right| ^ 2 = P_T = \exp{ \left\{ -2 \alpha a \right\} } \)
          \item Penetrazione media \( x_p = \dfrac{\hbar}{\sqrt{2 m (V_0 - E)}} = \dfrac{1}{\alpha} \)
          \item \textbf{L'approssimazione è valida se e solo se} \( \alpha a \gg 1 \)
        \end{itemize}
  \item Approssimazione di Fowler–Nordheim
        \begin{itemize}
          \item \textbf{Caso particolare: barriera triangolare} \( P_T = \exp\left\{ - \dfrac{4}{3} \dfrac{\sqrt{2 m}}{\hbar}  \dfrac{\Phi ^ {\sfrac{3}{2}}}{qF} \right\} \)
        \end{itemize}
\end{itemize}

\subsection{Incidenza}
\begin{itemize}
  \item Coefficiente di riflessione \( R = \left( \dfrac{k_1 - k_2}{k_1 + k_2} \right) ^ 2 \)
  \item Coefficiente di trasmissione \( T = \left( \dfrac{2 k_1}{k_1 + k_2} \right) ^ 2 = 1 - R ^ 2 \)
\end{itemize}

\subsection{Buca di potenziale}

\subsubsection{A pareti infinite}
\begin{itemize}
  \item Autovalori \( E_n = \dfrac{h^2}{8 m a ^2} n^2 \), spaziatura \( \propto n^2 \)
\end{itemize}

\subsubsection{A pareti finite}
\begin{itemize}
  \item Funzioni pari \( \tan\left( \dfrac{a}{2 \hbar} \sqrt{2mE} \right) = \sqrt{\dfrac{V_0 - E}{E}} \)
  \item Funzioni dispari \( \tan\left( \dfrac{a}{2 \hbar} \sqrt{2mE} \right) = - \sqrt{\dfrac{E}{V_0 - E}} \)
  \item La soluzione delle equazioni avviene per via grafica
\end{itemize}

\subsubsection{Parabolica}
\begin{itemize}
  \item Profilo di potenziale \( U = \dfrac{1}{2} \alpha x ^ 2 \)
  \item Pulsazione caratteristica \( \omega = \sqrt{\dfrac{\alpha}{m}} \), con \( alpha \) coefficiente del quadrato di \( x \)
  \item Autovalori \( E_n = \left( n + \dfrac{1}{2} \right) \, \hbar \omega \),  spaziatura \( \propto n \)
\end{itemize}

\newpage

\subsubsection{Coppie di buche}
\begin{itemize}
  \item Funzione degli autovalori \( \tan\left( k \dfrac{a}{2} \right) = - \dfrac{\hbar ^2 k}{m U_0} \)
  \item Proporzionalità della ddp \( \left| \psi \right| ^ 2 \propto \cos \left( \dfrac{E_2 - E_1}{\hbar} t \right) = \cos \left( 2 \pi \dfrac{E_2 - E_1}{h} t \right)  \)
        \begin{itemize}
          \item Oscillazione degli autovalori \( \omega = \dfrac{E_2 - E_1}{\hbar} \)
          \item Frequenza degli autovalori \( \nu = \dfrac{E_2 - E_1}{h} \)
        \end{itemize}
\end{itemize}

\section{Teoria semiclassica del trasporto}
\begin{itemize}
  \item Formula fondamentale \( \dfrac{dk}{dt} = \dfrac{F}{\hbar} \)
  \item Soluzione della formula fondamentale \( k = \dfrac{F}{\hbar} t + k_0 \)
  \item Velocità termica \( \displaystyle v_{th} = \sqrt{\dfrac{3 k T}{m}} \)
  \item Concentrazione intrinseca \( n_i = \sqrt{N_C N_V} \exp{\left\{ - \dfrac{E_C - E_V}{2 K T}\right\}} \)
\end{itemize}

\subsection{Tight Binding}
\begin{itemize}
  \item Massa efficace dell'elettrone \( m ^ \ast = \dfrac{ \mathfrak{F} }{a} = \dfrac{\hbar ^ 2}{\dfrac{\partial ^ 2 E}{\partial k ^ 2}} \)
  \item Relazione di dispersione \( E(k) = E_{0-} + 2 \gamma \cos(ka) \)
  \item Oscillazioni di Bloch \( \omega = \dfrac{a q F}{\hbar} \), \( \nu = \dfrac{a q F}{h} \)
  \item Libero cammino medio \( \lambda = v_{th} \cdot \tau_m \)
  \item Modello di Drude:
        \begin{itemize}
          \item Formula \( \dfrac{dk}{dt} + \dfrac{k}{\tau_m}= \dfrac{F}{\hbar} \) \textbf{vale solo per gli elettroni}
          \item Soluzione generale \( k = \dfrac{q \tau_m F}{\hbar} \left( 1 - e^{\sfrac{t}{\tau_m}} \right) \)
          \item Soluzione stazionaria \( \dfrac{\partial k}{\partial t} = 0 \rightarrow \bar{k} = \dfrac{q F}{\hbar} \tau_m \)
        \end{itemize}
  \item Masse DOS:
        \begin{itemize}
          \item Elettroni \( {m^\star} ^ {\sfrac{3}{2}} = g \cdot {m^\star_l} ^ {\sfrac{1}{2}} \, m^\star_t \)
          \item Lacune \( {m^\star} ^ {\sfrac{3}{2}} = {m^\star_{hh}} ^ {\sfrac{3}{2}} + {m^\star_{lh}} ^ {\sfrac{3}{2}} \)
        \end{itemize}
  \item Masse di conduzione:
        \begin{itemize}
          \item Elettroni \( \dfrac{\textnormal{n° di masse}}{m_c^\star} = \dfrac{\textnormal{n° di masse longitudinali}}{m_l^\star} + \dfrac{\textnormal{n° di masse trasversali}}{m_t^\star}\)
          \item Nel silicio \( m_c^\star = \dfrac{3}{\dfrac{1}{m_l^\star} + \dfrac{2}{m_t^\star}} \)
        \end{itemize}
\end{itemize}

\subsubsection{Semiconduttori}
\begin{itemize}
  \item Distribuzione di Fermi \( f(E) = \dfrac{1}{1+e^{\dfrac{E-E_F}{kT}}} \)
  \item Densità di stati di energia:
        \begin{itemize}
          \item Caso 1D \( \displaystyle g(E) = \dfrac{1}{\pi \hbar} \sqrt{\dfrac{2 m^\star}{E-E_F}} \)
          \item Caso 2D \( g(E) = \dfrac{m^\star}{\hbar^ 2 \pi} \)
          \item Caso 3D \( \displaystyle g(E) = \dfrac{(2 m^\star) ^ {\sfrac{3}{2}}}{2 \pi^2 \hbar^3} \sqrt{E-E_F} \)
        \end{itemize}
  \item Densità di portatori:
        \begin{itemize}
          \item Elettroni \( \displaystyle n = \int\limits_{E_F}^{\infty} g(E) f(E) dE \)
          \item Lacune \( \displaystyle p = \int\limits_{0}^{E_F} g(E) \left( 1- f(E) \right) dE \)
        \end{itemize}
  \item Energia media dell'elettrone \( \displaystyle \left<E\right> = \dfrac{1}{n} \int\limits_{E_F}^{\inf} E \cdot g(E) f(E) dE \), caso 3D \(\dfrac{3}{5} E_F \)
  \item Energia cinetica dell'elettrone \( E_k = \dfrac{\hbar^2 k^2}{2 m_n^\star} \)
\end{itemize}

\subsection{Weak binding}
\begin{itemize}
  \item Valore di aspettazione dell'energia al margine della funzione di Brillouin \( < E > = \dfrac{\hbar ^ 2}{2 m} \left(\dfrac{\pi}{a}\right) ^ 2 = E_n^+ - | u_n | \)
\end{itemize}

\subsubsection{Metalli}
\begin{itemize}
  \item Energia di Fermi \( E_F(T) = E_F(0 K) \left[ 1 - \dfrac{\pi ^ 2}{12} \left( \dfrac{kT}{E_F(0 K)}\right) ^ 2\right]\)
  \item Densità di portatori \textbf{approssimazione} \( n \approx \int\limits_{0}^{E_F} g(E) dE \)
\end{itemize}

\subsection{Formule valide sia per lacune che per elettroni}
\begin{itemize}
  \item Mobilità \( \mu = \dfrac{q \tau_m}{m^\star} \)
  \item Velocità di deriva \( v = \mu F \)
  \item Conducibilità \( \sigma = q n \mu \)
  \item Resistività \( \rho = \dfrac{1}{\sigma} \)
  \item Densità di corrente \( j = q n \mu F = \sigma F \)
\end{itemize}

\subsection{Livelli di energia}
\begin{itemize}
  \item Livello di Fermi \( E_f = \dfrac{ \left(3 \pi^2 n\right)^{\sfrac{2}{3}} }{2 m_n^*} \hbar ^ 2 \)
  \item Livello di energia intrinseco \( E_i = \dfrac{E_C + E_V}{2} + \dfrac{K T}{2} \ln\left( \dfrac{m_p^*}{m_n^*}\right) \)
\end{itemize}

\end{document}